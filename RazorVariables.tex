\section{Introduction}
The challanges at hadron collider experiments are many, one of the
most well known is the fact the the energy and momentum of the parton-parton
(hard) interaction is not know. This limitation, in addition with the
fact that a large number of the proposed extensions of the SM (e.g. SUSY) predict
new particles that are weakly interacting and therefore escape the
detector systems without a trace, have made events with large momentum
imbalance in the plane transverse to the beam a key signature to
search for BSM physics. The momentum imbalance is the transverse plane
(missing energy)
is quantified by $\ptvecmiss$  which is defined as 
\begin{equation}
\label{eq:ptmiss}
\ptvecmiss = -\sum\limits_{i=0}^{n_{pf}} \vec{p}_{T}^{\hspace{0.06cm}i},
\end{equation}
where $\vec{p}_{T}^{\hspace{0.06cm}i}$ is the measured transeverse
momentum of a PF candidate, and $n_{pf}$ is the total number of PF
candidates reconstructed.

Additionally, in SUSY models with R-parity conservation the originally
pair produced super-partners undergo a cascade decay and at least two
particles (the LSPs) will escape detection and therefore further reduce the
hability to fully reconstruct the event kinematics due to the lost
information. Subsequently, all these effects result in lost of
sensitivity as the discrimination power between a possible
signal and the SM background processes is reduced. In order to recover
sensitivity, different kinematic variables are employed which are
functions of the visible objects momenta and the $\ptvecmiss$. These
kinematic variables have been shown to improve signal to background
discrimination but are often model dependent. One example of such
variables arew the razor variables~\cite{rogan,razor2010}, which have been widely used by the
CMS collaboration to search for
SUSY~\cite{Chatrchyan:2014goa,Razor8TeV} and recently shown to have
good sensitivity for DM direct production at hadron
colliders~\cite{Fox:2012ee}. The razor variables: $\MR$ and $\RR^2$  provide an estimate of the
underlying mass scale of the event and a handle to significantly
supress SM backgrounds -- particularly QCD multiijet --,
respectively. 

Since the two searches for BSM physics presented in this thesis are
based on the razor variables, this Chapter describes its derivation
(see Section~\ref{razorVariables})
and their main features when searching for new physics (see Section~\ref{razorApp}).

\section{The Razor Variables}\label{razorVariables}
The razor variables were originally derived~\cite{rogan,razor2010} for
squark pair-production in the context of SUSY; this topology is
represented by the Feynman diagram shown in Figure~\ref{fig:squarkpair}, where
the proton-proton collision pair produces two squarks ($\tilde{q}_{1}\tilde{q}_{2}$) which
subsequently decay into a SM quark and the LSP ($\tilde{q}_{i}\rightarrow q_{i}\tilde{\chi}_{1}^{0}$).

\begin{figure}
 \centering
\includegraphics[width=0.4\textwidth]{RazorVariables/T2qq.pdf}
 \caption{Feynman diagram for squark pair-production.\label{fig:squarkpair}}
\end{figure}
One interesting quantity that provides access to the mass scale of the
SUSY particles is the magnitude of the 3-momentum of the quark in the
rest frame of the squark, it is actually more convenient to write down
twice this quantity:
\begin{equation}
\label{eq:pq}
2|\vec{p}^{\hspace{0.06cm}q_{i}}| =
2|\vec{p}^{\hspace{0.06cm}\tilde{\chi}_{1}^{0}}| =
\frac{\sqrt{m^{4}_{\tilde{q}} -
2m^{2}_{\tilde{q}}m^{2}_{\tilde{\chi}_{1}^{0}} +
m^{4}_{\tilde{\chi}_{1}^{0}} - 2m^{2}_{q}m^{2}_{\tilde{\chi}_{1}^{0}}
- 2m^{2}_{q}m^{2}_{\tilde{q}} + m_{q}^{4}}}{m_{\tilde{q}}},
\end{equation}
 where ..... Eq.~\ref{eq:pq} can be further simplified if the SM
 quarks are assumed massless -- which is mostly accurate with the
 exception of the top-quark. This simplification is also used to
 define :
\begin{equation}
\label{eq:Mdelta}
m_{\Delta} = 2|\vec{p}^{\hspace{0.06cm}q_{i}}| =
\frac{m^{2}_{\tilde{q}} -
m^{2}_{\tilde{\chi}_{1}^{0}}}{m_{\tilde{q}}}
\end{equation}

\section{Application of the Razor Variables to Search for BSM Physics}\label{razorApp}

