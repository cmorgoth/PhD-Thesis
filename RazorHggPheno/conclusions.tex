\section{Discussion and summary}
\label{sec:conclusionspheno}

In this paper, we proposed two simplified models of bottom squark
pair production for use in the interpretation of an excess observed
by CMS in a search for SUSY in $\PH+$jets events using razor variables at $\sqrt{s}=8\TeV$~\cite{RazorHgaga}. In model A, we considered the
asymmetric production of a $\sbottom_2
\sbottom_1$ pair, with the $\sbottom_1\to\chizone$, $\sbottom_2\to\cPqb
\chiztwo$, and $\chiztwo \to H \chizone$, where $\chizone$ is a
neutralino LSP and we fix the mass splitting $m_{\chiztwo}-m_{\chizone}=130\GeV$. In model B, we considered the symmetric production of a
$\sbottom_1\sbottom_1$ pair, with $\sbottom_1 \to \cPqb \chiztwo$,
$\chiztwo \to \PH \chizone$, and
$m_{\chiztwo}-m_{\chizone}=130\GeV$. 

We scanned the bottom squark masses
for a fixed LSP mass of $m_{\chizone}=100\GeV$ for both models and
quantified the agreement with the data. We found
the excess observed in data is broadly consistent with both models,
with the largest signal significance being $1.8\sigma$
corresponding to model B with $m_{\sbottom_1}=500\GeV$,
$m_{\chiztwo}=230\GeV$, and $m_{\chizone}=100\GeV$. Following this study, model B used by the CMS collaboration to interpret the
results of the updated 13\TeV search for SUSY in the same
channel~\cite{CMS-PAS-SUS-16-012}, which also exhibits an excess
possibly consistent with the model.
%Interestingly, following this study, model B used by the CMS collaboration to interpret the
%results of the updated 13\TeV search for SUSY in the same channel~\cite{CMS-PAS-SUS-16-012}, where the
%largest signal significance was found to be $2.1\sigma$ for the same masses of $m_{\sbottom_1}=500\GeV$, $m_{\chiztwo}=230\GeV$, and
%$m_{\chizone}=100\GeV$.