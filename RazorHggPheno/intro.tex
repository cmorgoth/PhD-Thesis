\section{Introduction}
 
The ATLAS and CMS collaborations intensively searched for SUSY production in the data collected at a center-of-mass energy $\sqrt{s}=8 \TeV$ in 2012. A large part of the searches focused on SUSY models with conserved R-parity, for which the lightest SUSY particle (LSP) is stable. The LHC is particularly sensitive to the production of SUSY partners charged under QCD (squarks and gluinos), given the large production cross section in proton collisions. Given the strong bounds on generic SUSY models derived with $\sqrt{s}=7 \TeV$ data, ATLAS and CMS moved the focus of their SUSY searches to the so-called \textit{natural} SUSY models~\cite{Papucci:2011wy}. In its minimal realization, a natural SUSY spectrum is composed of the minimum set of SUSY partners needed to protect the mass of the Higgs ($\PH$) boson from quantum corrections: a gluino, one bottom squark, two top squarks, and three higgsinos (two neutral and one charged). This SUSY scenario results in events with multiple top and bottom quarks, produced in association with missing transverse energy \met. No evidence for the production of such particles was found, pushing the allowed mass range for gluinos and top
squarks above $\sim 1600 \GeV$ and $\sim 700 \GeV$, respectively, for a low-mass neutralino LSP and largely independent of the top squark and
gluino branching ratios (see for instance Ref.~\cite{razor8TeV,CMS-PAS-SUS-15-004}). 

In a few cases, a data yield above the expected background was
observed for certain signal regions, for example, in the case of the
\emph{edge} dilepton analysis by CMS~\cite{CMSedge} and the SUSY
search in $\PZ+$jets events by ATLAS~\cite{ATLASZpeak}. These excesses
correspond to, respectively, $\sim 2.4\sigma$ and $\sim 3.0\sigma$ of
local significance, which are reduced after accounting for the
look-elsewhere effect (LEE). Several interpretations of these results
were given in the literature~\cite{Theory1,Theory2,Theory3,Theory4,Theory5,Theory6},
mainly related to the electroweak production of SUSY particles with long decay chains. 

Here we discuss another interesting excess, observed in a search
for electroweak SUSY partners in $\PH (\Pgg\Pgg)+ \geq 1$~jet events by the CMS
collaboration performed at $8\TeV$~\cite{RazorHgaga}. The analysis uses the diphoton
invariant mass \mgaga to select events with a $\PH$-like
candidate. The nonresonant (mostly QCD
diphoton production) and resonant (standard model $\PH(\Pgg\Pgg)$
production) backgrounds are estimated using the \mgaga sidebands in data
and the Monte Carlo simulation, respectively. The background prediction is performed as a
function of the razor variables $\MR$ and $\Rtwo$ in five mutually
exclusive \emph{boxes}, targeting different final states:
high-$\pt$ $\PH (\Pgg\Pgg)$ (\texttt{HighPt} box), $\PH
(\Pgg\Pgg)+\PH (\bbbar)$ (\texttt{Hbb} box), $\PH
(\Pgg\Pgg)+\PZ (\bbbar)$ (\texttt{Zbb} box), and low-$\pt$ $\PH
(\Pgg\Pgg)$ with high- and low-resolution photons
(\texttt{HighRes} and \texttt{LowRes} boxes, respectively). Five events are
observed in one ($\MR$, $\Rtwo$) bin of the \texttt{HighRes} box, compared
to less than one expected background event. This corresponds to a
local significance of $2.9\sigma$, reduced to $1.6\sigma$ after the
LEE. 

In this paper, we discuss a possible interpretation of this search in
terms of SUSY models with light quarks. We emulate this CMS analysis
to derive bounds on squark production. Since the analysis does not
require or veto jets originating from \cPqb-quarks (\cPqb-jets), the results
apply to bottom-squark production in natural SUSY models. 

Recently, an updated search was performed with data collected at
$13\TeV$~\cite{CMS-PAS-SUS-16-012}, which exhibits a similar excess of
$2.5\sigma$ local significance, reduced to $1.4\sigma$ after the
LEE. Model B proposed in this paper was also used for the interpretation of the results. 

%The paper is organized as follows: in section~\ref{sec:models} we introduce the benchmark SUSY models considered in this study; section~\ref{sec:gensim} describes the event generation and the detector emulation; the analysis setup is described in section~\ref{sec:analysis}. Results and conclusions are given in sections~\ref{sec:results}~and~\ref{sec:conclusions}, respectively.