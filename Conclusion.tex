\chapter{Conclusion and Discussion}

We presented a summary of the SM most important ingredients, we also
outlined some of the issue with this theory, such as the hierarchy
problem, the non-unification of the gauge coupling at the Planck
scale, and the lack of a suitable dark matter candidate, among others. We also
presented a review of the astronomical and astrophysical evidence for
dark matter as well as the status of searches for WIMP dark matter. The problems
with the SM, the evidence for dark matter and it possible production
in particle colliders, motivated the searches for beyond the standard model physics
presented in this thesis.

We review the most important aspects of the LHC machine and the
different detectors operating at the Compact Muon Solenoid experiment,
these include, the CERN accelerator facility, the LHC superconducting
magnets, the tracker pixel and strip detectors, the electromagnetic
calorimeter, the hadronic calorimeter, the muon system, the
superconducting solenoid, and the trigger system. All of these
ingredients are essential for carrying out the searches for beyond the
standard model physics presented in this thesis. We also presented a
review of the properties of the
razor variables for discovering new physics. 

We presented a search for particle dark matter in events with two or
more jets using $20 \mathrm{fb}^{-1}$ of $\sqrt{s} = 8\TeV$ data. This
search used the razor kinematic variable to discriminate signal from
background events. We observed good agreement between the observations
and the estimated background yield, therefore we set 90\% CL. limits
in the cutoff scale ($\Lambda$) of the vector and axial vector
effective field theory operators considers. We found that the sensitivity of the search was
very competitive with respect to that of the standard monojet
searches, setting a limit on $\Lambda ~ 1\TeV$, and using a phase
space not yet explored thus improving the CMS sensitivity for particle
dark matter production. The final results were interpreted as a
DM-nucleon cross-section limit as a function of the DM candidate mass
and compared to the 90\% CL. from direct detection experiments. The
LHC results are very competitive in the spin-dependent case and at
masses below a few \GeV.

We also presented a search for anomalous Higgs boson production using
$15.3 \mathrm{fb}^{-1}$ of $\sqrt{s} = 13\TeV$ data.
This analysis selects events with a Higgs boson in association with
jets, where the Higgs candidate decays into two
photons in the central region of the CMS detector. This analysis uses
the razor variables to discriminate between signal and background. The
are several a total of 14 $\mathrm{M_{R}}-\mathrm{R}^{2}$ search
regions and we observed a 2.4$\sigma$ excess in one of the bins with
the highest values of $\mathrm{M_{R}}-\mathrm{R}^{2}$ and where the
Higgs candidate has a transverse momentum larger that 110\GeV. Another
excess of events was observed in the 8\TeV analog of this analysis but
in a different $\mathrm{M_{R}}-\mathrm{R}^{2}$. Nevertheless, this
excess has to be followed closely and confirmed or disproved with a
larger integrated luminosity.


We dedicated an entire part of this thesis to present the detector
research and develop towards a precision timing calorimeter. This
worked was carried out with Fermilab and Caltech collaborators where
we studied different calorimeter prototypes equipped with precision
timing capabilities. We studied LYSO-based sampling calorimeters,
``shashlik'' sampling calorimeters, multichannel plates as the active
element of a sampling calorimeters, silicon detectors as the active
element of a sampling calorimeter, and finally we studied one of the
modules proposed to be used in the Phase-II upgrade of the CMS
electromagnetic calorimeter. The prototypes where tested at the
Fermilab Test Beam Facility and at CERN with electron and proton
beam. We found time resolution of the order of 10-50\unit{ps}
depending on the calorimetric device. These results are very
encouraging, particularly in light of its application to alleviate the
detrimental effects of pileup on particle reconstruction and
identification at the High Luminosity LHC.

We have also presented in the appendices a search for beyond the
standard model physics in high-mass diphoton resonances. This search
observed a $2.9\sigma$ excess at an invariant mass of about 750\GeV in the data collected during 2015 and
drew significant attention from the community. We presented in this
thesis an update of this analysis with approximately 5 times the
integrated luminosity, where the 750\GeV excess was found to be
greatly disfavored. The CMS collaboration decided that an independent
analysis should be carried out given the importance of this
analysis. This second parallel analysis was carried out in an
independent fashion and the two analyses were found to be compatible,
thus providing confidence in the observed results. No significant
excess of events over the SM background was found and 95\% CL. limits
were place in the context of a massive spin-0 and a spin-2
Randall-Sundrum resonance with various experimental widths.
Also in the appendices is a reinterpretation of the search for
anomalous Higgs boson production using razor variables carried out by
CMS at 8\TeV. The reinterpretation includes two model of bottom squark
pair production that decays into neutralinos and the Higgs boson. The
search studied seems to have a good sensitivity for this particular
supersymmetric scenario excluding the bottom squark up to a mass of
about 350\GeV. It is also of note that this event topology seems to be
consistent with the excess of events reported by the CMS analysis
where the preferred bottom squark mass was found to be around 500\GeV.

Despite the null results regarding the discovery of new physics, the
LHC experiments have covered a large amount of well motivated searches for
BSM physics with a few interesting excesses to be followed closely. In particular this thesis presented a pioneer effort to search
for particle dark matter production at the LHC, which has now become
one of the most widely covered topic at the LHC experiments, including
final state with a lepton, photons, W/Z boson, and recently Higgs
boson. Another direction that we have studied is the more model
independent search for anomalous Higgs production, which has been
enabled by the measurements of the SM Higgs properties. This search is
very relevant since the Higgs is most likely involved in beyond the SM
process that could enhance its production or decay rate. A possible
extension of this analysis is to add searches targeting different
Higgs final states as well as to search for associated production of
HW, HZ, and HH. The later, will be realized by the large integrated
luminosities to be collected in the high luminosity run of the LHC. We
also presented a very model independent search for high-mass diphoton
resonances which will remain very important for the LHC program as well
as other resonance searches that will become feasible, again, as more
integrated luminosity is collected. 

We want to emphasize the importance of developing new detector
technologies that will enable us to answer the most important puzzles
in nature today. In particular the pursue of sub-10\unit{ps} precision timing devices,
including calorimeters, is an area that shows a lot of potential
applications for particle collider experiments and that has been
proven to be readily available.

Finally, with larger integrated luminosities and possible increases in
beam energies -- either by an upgrade of the LHC magnets or by the
construction of new particle colliders -- we will continue search for
new phenomena by probing rare processes and exploring new phase space.
